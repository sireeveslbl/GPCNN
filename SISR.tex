\documentclass{article}%{elsarticle}

\usepackage{lineno,hyperref}
\usepackage{color}
\usepackage{amsfonts}
\usepackage{amsmath}
\usepackage{amssymb}
\usepackage{bm}
\usepackage{mathabx} 
\usepackage{multirow}
\usepackage{subfig}
\usepackage{graphicx}
\usepackage{rotating}
\usepackage{listings}
\usepackage{tikz, pgf}
\usepackage{tikz-3dplot}

\def\LW{\dimexpr.25\linewidth-.5em}

\makeatletter
\def\@xfootnote[#1]{%
  \protected@xdef\@thefnmark{#1}%
  \@footnotemark\@footnotetext}
\makeatother

\input macros.tex

\modulolinenumbers[5]

\begin{document}

%%%%%%%%%%%%%%%%%%
%% TITLE 
%%%%%%%%%%%%%%%%%%
\title{A Hybrid Gaussian Process Convolution Neural Network Approach to Single Image Super Resolution}
%% Group authors per affiliation:
\author{Steven I. Reeves \and Dongwook Lee}
%\author[$\dagger$]{Dongwook Lee}%\corref{mycorrespondingauthor}}
%\cortext[mycorrespondingauthor]{Corresponding author}
%\ead{dlee79@ucsc.edu}


\maketitle

\begin{abstract}
Single image super-resolution is a notoriously difficult ill-posed problem. We present a novel hybrid Gaussian Process - Deep
Convolutional Neural Network approach to solving this problem. 
\end{abstract}

%\begin{keyword}
%Gaussian processes; 
%\end{keyword}

%\end{frontmatter}

%\linenumbers


%%%%%%%%%%%%%%%%%%%%%%%%%%%%%%%%%%%%%%%%%%%%%%%%%%
%%%%%% INTRODUCTION %%%%%%%%%%%%%%%%%%%%%%%%%%%%%%%%%%%
%%%%%%%%%%%%%%%%%%%%%%%%%%%%%%%%%%%%%%%%%%%%%%%%%%

\section{Introduction}
\label{sec:introduction}

We propose a new method for upsampling low resolution images from a single base image alone. Single Image Super Resolution (SISR),
is a difficult problem in the field of image processing and computer vision. 

%%%%%%%%%%%%%%%%%%%%%%%%%%%%%%%%%%%%%%%%%%%%%%%%%%
%%%%%% Gaussian Processes %%%%%%%%%%%%%%%%%%%%%%%%
%%%%%%%%%%%%%%%%%%%%%%%%%%%%%%%%%%%%%%%%%%%%%%%%%%
\section{Gaussian Process Modeling}
\label{sec:GP}
The method we are presenting is based on Gaussian Process Modeling, and in this
section we give a brief overview on constructing a Gaussian Process Model. Gaussian Processes
are a family of stochastic processes in which any finite collection of random variables sampled
from this process are joint normally distributed. In a more general sense, Gaussian Processes sample functions
from an infinite dimensional function space. The interpolating routine described in detail in section~\ref{sec:method},
will be drawn from a data-informed distribution space trained on the low resolution data.

To construct a Gaussian Process model, one needs to specify a \textit{prior probability distribution} for
 the function space. Samples (function values evaluated at known locations)
are then used to update this prior probability distribution, and using Bayes' Theorem a \textit{posterior probability
distribution} is generated given the prior and the samples.

\subsection{A statistical introduction to Gaussian Processes}

The construction of the posterior probability distribution over the function space is the heart of
Gaussian Process Modeling. We can draw functions from this data adjusted space to generate an interpolating model.
 Specifically, the posterior may be used to probabilistically predict the value of a function at points where the
function has not been previously sampled.

A Gaussian Process is a collection of random variables, in which any finite collection has a joint Gaussian distribution
\cite{Rasmussen2005}\cite{pattern}.
GPs can be fully defined by two functions:
 a mean function $\bar{f}(\mathbf{x}) = \mathbb{E}[f(\mathbf{x})]$ and a
 covariance function that generates a symmetric, positive-definite kernel
 $K(\mathbf{x}, \mathbf{y}): \mathbb{R}^N\times\mathbb{R}^N \to \mathbb{R}$.

We denote functions $f$ drawn from a GP with mean function $\bar{f}(\mathbf{x})$ and covariance
$K(\mathbf{x}, \mathbf{y})$ as $f\sim \mathcal{GP}(\bar{f}, K)$. Analogous to finite-dimensional
distributions we write the covariance as
\begin{equation} 
K(\mathbf{x}, \mathbf{y}) = \mathbb{E}\left[\left(f(\mathbf{x}) - \bar{f}(\mathbf{x})\right)
\left(f(\mathbf{y}) - \bar{f}(\mathbf{y})\right)\right]
\end{equation}
where $\mathbb{E}$ is with respect to the GP distribution.

One controls the GP by specifying both $\bar{f}(\mathbf{x})$ and $K(\mathbf{x}, \mathbf{y})$,
typically as some hyper-parameterized functions. These hyper-parameters allow us to give the
"character" of functions generated by the posterior (i.e. length scales, differentiability).
Suppose we have a given GP, and $N$ locations $\mathbf{x}_i, i = 1, \dots, N$ at which we collect
samples $f(\mathbf{x}_i$, then we can calculate the likelihood $\mathcal{L}$ -- the probability of
the data given the GP model. Let
$\mathbf{f} = \left[f(\mathbf{x}_1, \dots, f(\mathbf{x}_N) \right]^T $ then the likelihood is
\begin{equation} 
\mathcal{L} \equiv P(\mathbf{f} | \mathcal{GP}(\bar{f}, K)) = (2\pi)^{-N/2} \det |\mathbf{K}|^{-1/2} 
\exp\left[-\frac{1}{2}\left(\mathbf{f} - \bar{\mathbf{f}}\right)\mathbf{K}
\left(\mathbf{f} - \bar{\mathbf{f}}\right)\right]
\label{eq:likely}
\end{equation}
where $\mathbf{K}$  is a matrix generated by
$K_{i,j} = K(\mathbf{x}_i, \mathbf{x}_j)$, $i, j = 1,\dots, N$,
and $\bar{\mathbf{f}} = [\bar{f}(\mathbf{x}_1), \cdots \bar{f}(\mathbf{x}_N)]$. Using these samples we can make a
probabilistic statement about the value of the function $f \sim \mathcal{GP}(\bar{f}, K)$
at a new point $\mathbf{x}_*$. That is, we can model the value of $f(\mathbf{x}_*)$ using this GP model. This
is especially import for Adaptive Mesh Refinement, as we need to construct data at a finer resolution.

An application of Bayes' Theorem along with the conditioning property, directly onto the joint Gaussian prior
given the samples $\mathbf{f}$ give the posterior distribution of the predicted value, $f_*$ given $\mathbf{f}$.
\begin{equation} 
P(f_* | \mathbf{f}) = (2\pi U^2)^{-1/2} \exp\left[- \frac{(f_* - \bar{f}_*)^2}{2U^2}\right].
\end{equation}
Perhaps more important for this application, the posterior PDF gives a new \textit{posterior mean function}
\begin{equation}
\tilde{f}_* \equiv \bar{f}(\mathbf{x}_*) + \mathbf{k}_*^T\mathbf{K}^{-1}\cdot(\mathbf{f} - \bar{\mathbf{f}})
\label{eq:mean}
\end{equation}
and \textit{posterior covariance}
\begin{equation} 
U^2 \equiv k_{**} - \mathbf{k}_*^T\mathbf{K}^{-1}\cdot\mathbf{k}_*.
\end{equation}

\subsection{Choice of Covariance Kernel} 
The choice of covariance kernel can have a dramatic effect on the model 
constructed. Two common kernels that we investigate are the Squared-Exponential 
kernel (see equation~\ref{eq:sqrexp}) and the Matern kernel(see equation~\ref{eq:matern}).  

The squared exponential kernel essentially assumes that all functions drawn from a Gaussian Process generated with it 
are $C^\infty$ continuous.  
\begin{equation}
K(\mathbf{x}, \mathbf{y}) = \Sigma^2 \exp{\left(-\frac{1}{2l^2}||\mathbf{x} - \mathbf{y}||^2\right)}
\label{eq:sqrexp}
\end{equation}

\begin{equation}
\label{eq:matern}
\end{equation}

%%%%%%%%%%%%%%%%%%%%%%%%%%%%%%%%%%%%%%%%%%%%%%%%%
%%%%%%% Algorithm %%%%%%%%%%%%%%%%%%%%%%%%%%%%%%%%%%%%%
%%%%%%%%%%%%%%%%%%%%%%%%%%%%%%%%%%%%%%%%%%%%%%%%%%



%%%%%%%%%%%%%%%%%%%%%%%%%%%%%%%%%%%%%%%%%%%%%%%%%
%%%%%%% References %%%%%%%%%%%%%%%%%%%%%%%%%%%%%%%%%%%%%
%%%%%%%%%%%%%%%%%%%%%%%%%%%%%%%%%%%%%%%%%%%%%%%%%%
\newpage

\section*{References}
\label{sec:references}
\bibliographystyle{siam}
\bibliography{mybibfile}




\end{document}
%%% Local Variables:
%%% mode: latex
%%% TeX-master: t
%%% End:
